\chapter{Introdução}
\label{cap:introducao}
A maioria dos problemas que exigem classificação, reconhecimento, controle, identificação, previsão, entre outros, onde a utilização de inteligência é exigida, o uso das Redes Neurais Artificiais(RNA) tem se mostrado uma poderosa ferramenta. Segundo \citeonline{silva2010redes} RNA vem sendo utilizada em diversas áreas, desde problemas de otimização computacional à diagnóstico de doenças. 

Em RNA exitem algumas arquiteturas onde cada uma objetiva solucionar um tipo de problema, em termos de classificação de padrões podemos citar Mulayer Perceptron(MLP), Radial Basis Function(RBS), Adaptive Resonance Theory(ART) e Kohonen. Essas redes são divididas em redes com treinamento supervisionado e redes com treinamento não supervisionado.

 As redes com treinamento supervisionado tem como característica o fato de se ter disponível, considerando cada amostra dos sinais de entrada, as respectivas saídas desejadas através de uma tabela de atributos e valores, pois é a partir dessas informações que as estruturas formularão um padrão sobre como as saídas devem ser.
 
 Diferentemente das redes com treinamento supervisionados temos as redes ...




\section{Motivação}
\label{sec:motivacao}


\section{Objetivos}
\label{sec:objetivos}

\subsection{Objetivo Geral}
\label{sec:objetivo-geral}

\subsection{Objetivo Específico}
\label{sec:objetivo-específico}
