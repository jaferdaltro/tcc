\chapter{Introdução}
\label{cap:introducao}
A maioria dos problemas que exigem classificação, reconhecimento, controle, identificação, previsão, entre outros, onde a utilização de inteligência é exigida, o uso das Redes Neurais Artificiais(RNA) tem se mostrado uma poderosa ferramenta. Segundo \citeonline{silva2010redes} RNA vem sendo utilizada em diversas áreas, desde problemas de otimização computacional à diagnóstico de doenças. 

Em RNA exitem algumas arquiteturas onde cada uma objetiva solucionar um tipo de problema, em termos de classificação de padrões podemos citar Mutilayer Perceptron(MLP), Radial Basis Function(RBS), Adaptive Resonance Theory(ART) e Kohonen. Essas redes são divididas em redes com treinamento supervisionado e redes com treinamento não supervisionado.

 As redes com treinamento supervisionado tem como característica o fato de se ter disponível, considerando cada amostra dos sinais de entrada, as respectivas saídas desejadas através de uma tabela de atributos e valores, pois é a partir dessas informações que as estruturas formularão um padrão sobre como as saídas devem ser, ajustando assim os pesos sinápticos e os limiares de cada neurônio.
 
 Diferentemente das redes com treinamento supervisionados, quando se é aplicado um algoritmo de treinamento há as saídas desejadas para os respectivos resultados, dessa forma, a rede deve se organizar de modo a identificar os subconjuntos que contenha similaridades, fazendo com que os pesos sinápticos e os limiares dos neurônios se ajustem de forma que reflitam a representação interna da própria rede.


\section{Objetivos}
\label{sec:objetivos}

\subsection{Objetivo Geral}
\label{sec:objetivo-geral}

O objetivo central deste trabalho consiste em realizar um estudo comparativo sobre o desempenho das RNA que possuem treinamenento supervisionado e 	RNA que possuem treinamento não supervisionado considerando cinco bases de dados:

\begin{enumerate}
	\item Classificação de lírios(flor) em três espécies
	\item Avaliar se um cancêr é maligno ou benigno
	\item Classificação de cogumelos entre venenosos ou não
	\item Fraudes em cartões de crédito
	\item Diagnóstico de Diabetes	
\end{enumerate}


\subsection{Objetivo Específico}
\label{sec:objetivo-específico}
Em termo de objetivos específicos pretende-se:
	\begin{alineas}
		\item Realizar testes para clusterizar cada base de dados com as redes MLP, RBS, ART e Kohonen.
		\item Anotar os resultados referente as taxas de erros e acurácia.
		\item Comparar os resultados afim de determinar qual tipo de treinamento obtém os melhores resultados diante dos dados apresentados.
	\end{alineas}
